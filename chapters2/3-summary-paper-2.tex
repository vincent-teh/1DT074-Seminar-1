\section{Summary Paper 2}%
\label{sec:Summary Paper 2}
The main focus of paper \cite{C2} is to point out the problem of Kubernetes storing all cluster state in \textit{etcd}, which causes high latency and decreases throughput with large clusters. This reduces Kubernetes suitability for the edge. The edge is of upmost importance in this paper, without really explaining what it is. Simply put, edge as a distributed architecture, moving computing from a centralized data center / cloud with high bandwidth, low latency conections to the distributed network making it decentralized \cite{DBigelow_2021}.\\ Focusing on \textit{etcd}, paper \cite{C2} explains that the provided strong consistency might be better abandoned in the context of edge. After a brief explanation of \textit{etcd} and how it is based on the \textit{Raft} consensus protocol, the authors focus on its performance during scaling, showing that write latency increases heavily with more cluster nodes. The throughput is also impacted in large etcd clusters, regardless of request type \cite{C2}. \\
With such results, a solution is proposed in the \textbf{etcd API} and \textbf{Lazy syncing}. Enabling reads and writes to a single node without immediate communication with other nodes enables fast responses to the API server even on larger scales but introduces conflicts in stored data. These conflicts are to be resolved by \textbf{Conflict-Free Replicated Datatypes} (CRDTs) eventually, using Lazy syncing to do so. CRDTs come in two varieties: state-based and operation-based \cite{C2}. These enable conflict resolution upon syncing with other nodes providing eventual consistency of the Kubernetes resource objects, however the design of CRDTs is not the focus of this paper.\\
The proposed solution is based on the idea of probabilistically bounded staleness showing that an eventually consistent system can often still present the latest updates to data \cite{C2}, basically abandoning strong consistency of Kubernetes provided by \textit{etcd} it favour of performance on edge by improving scalability.