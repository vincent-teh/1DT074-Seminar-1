\section{Similarities and Differences}

The three papers collectively address Kubernetes optimizations across different layers, targeting performance, scalability, and suitability for edge and IoT environments. All emphasize empirical evaluation: \cite{C1} and \cite{C3} benchmark existing components (CNI plugins and lightweight CODs, respectively), while \cite{C2} combines performance analysis of etcd with a proposal for architectural redesign. A shared theme is the critical examination of Kubernetes subsystems—networking (\cite{C1}), control-plane storage (\cite{C2}), and orchestration frameworks (\cite{C3})—to address inefficiencies in distributed systems.

\cite{C1} and \cite{C3} focus on aiding practitioners in selecting optimal solutions. Both employ granular performance metrics, such as CPU cycles per packet (\cite{C1}) and control-plane latency (\cite{C3}). In contrast, \cite{C2} identifies etcd’s strong consistency as a scalability bottleneck for edge deployments and proposes replacing it with an eventually consistent CRDT-based store. This positions \cite{C2} as a forward-looking architectural critique, whereas \cite{C1} and \cite{C3} are grounded in comparative analysis of existing tools.

\cite{C2} and \cite{C3} share a focus on edge computing but diverge in scope. \cite{C2} targets control-plane scalability through etcd replacement, addressing availability and latency in large clusters. \cite{C3} evaluates end-to-end performance of lightweight CODs, emphasizing resource constraints typical of edge nodes. \cite{C1}, while not explicitly edge-focused, provides insights relevant to edge networking, such as the impact of CNI choices on pod communication latency. Methodologically, \cite{C1} and \cite{C2} analyze low-level interactions: \cite{C1} dissects kernel-space overheads (e.g., Netfilter, eBPF), while \cite{C2} benchmarks etcd’s write latency and throughput. \cite{C3} adopts a higher-level approach, integrating system-wide metrics (e.g., CPU, disk usage) and application-layer performance (CoAP workloads).