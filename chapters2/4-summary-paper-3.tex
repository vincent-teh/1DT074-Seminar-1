\section{Summary Paper 3}%
\label{sec:Summary Paper 3}

The paper \cite{C3} presents a \textbf{comparison of four major Container Orchestration Distributions} called CODs (k0s, k3s, Microk8s, and Microshift) which are lightweight versions of Kubernetes optimized for edge computing and IoT environments. The authors aim to facilitate the choice of a COD depending on its environment's constraints.\\

The comparison is based on \textbf{three metrics} that all might impact user experience, namely the hardware resource usage, the control-plane performance and the data-plane performance. The measuring tools used are Sysstat, K-Bench and CoAPthon respectively. \\
The \textbf{resource utilization} includes CPU, memory, disk, and network usage measurements and determines the efficiency of resource allocation. It is important to note that by nature, the CODs consume fewer resources than full-scale Kubernetes since they are designed to be integrated in resource-constrained devices. \\
The \textbf{control-plane performance} demonstrates the CODs' management of clusters by measuring the latency and throughput of API operations. It covers both the Deployment API and the Kubernetes objects which are declarative representations of a desired state within clusters, such as Pods or Service objects.\\
Finally, the \textbf{data-plane parameter} focuses on CODs' management of application traffic by measuring the throughput and latency of two types of services (ClusterIP and NodePort) in two deployment scenarios (in master or worker node). \\

The authors propose \textbf{mathematical models} for each of these performance factors. The input parameters (for example latency of actions or object type) are weighted and normalized on $0-1$ scale from best to worst performance. \\
As a result, \textbf{k3s demonstrates the best performance} for the three factors and seems to be the best choice overall. \textbf{k0s performs well for high throughput environments as well as Microshift} but the latter lacks of multi-node support and exhibits a higher resource usage. \textbf{Microk8s shows the least performance} and is not recommended. \\
The authors discuss the \textbf{limitations} of their study, pointing out that research under constrained environments may not reflect real-life conditions such as geographically distributed nodes, and that mathematical models imply bias and simplifications. 
