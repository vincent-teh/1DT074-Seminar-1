\section{Key Insights from the Papers}%
\label{sec:Key Insights from the Papers}

All three papers are critical of and work to reduce the \textbf{ossification }of the Internet. In particular, the first paper \cite{langley2017quic} introduces the  new QUIC transport protocol, despite the challenges caused by ossification. However, since QUIC is built on top of UDP, it does not challenge the ossification to the extent that entirely new protocols can be established. The second paper \cite{grinnemo2016towards} provides a more radical way to combat the ossification of the Internet, by decoupling applications from their choice of transport protocol. The third paper \cite{kosek2021beyond} reaps the benefits of the anti-ossification measures of the QUIC protocol sowed in the first. It showcases the wide variety of plugins and extensions to the functionality of the QUIC protocol that have been and are being developed, all of which would have been impossible if QUIC hadn't been designed with anti-ossification measures and with extendability in mind. 

All three papers also emphasize the importance of \textbf{reducing latency}, each addressing the issue from different angles. The first paper \cite{langley2017quic} introduces QUIC with low latency as one of its key design goals. QUIC achieves this through features like 0-RTT handshakes, which allow data to be sent immediately on repeat connections, and stream multiplexing, which eliminates HOL blocking. These innovations significantly reduce latency compared to traditional TCP-based protocols, particularly for short-lived connections and in high-latency environments. The second paper \cite{grinnemo2016towards} takes a broader approach with its NEAT architecture, which allows applications to express their latency requirements through a richer API. This enables the transport layer to dynamically select the most appropriate protocol and configuration to minimize latency based on real-time network conditions and application needs. The third paper \cite{kosek2021beyond} builds on QUIC's low-latency foundation by exploring extensions like unreliable datagrams and multipath QUIC, which further enhance performance for latency-sensitive applications such as real-time communication and video streaming.