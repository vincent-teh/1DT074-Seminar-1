\section{Summary Paper 3}%
\label{sec:Summary Paper 3}
The Internet Engineering Task Force (IETF) has released new standardizations on top of QUIC v1 in the form of extensions and the use of QUIC as a substrate protocol.

Among the examples of QUIC extensions is the \textbf{Unreliable Datagram Extension}, which allows QUIC to handle unreliable datagrams, enhancing use cases where reliability is not a strict requirement. 
Another notable extension is \textbf{QUIC-LB}, which enables QUIC to maintain its connection ID during network migrations, thereby preserving connection continuity. 
Additionally, the \textbf{Multipath Extension for QUIC} allows QUIC to preserve its single-path semantics while simultaneously utilizing multiple paths for a single connection. 
Furthermore, \textbf{Greasing the QUIC bit} is an essential feature designed to randomize the QUIC bit used to identify the protocol, ensuring that endpoints can still identify received packets as QUIC packets. 
This feature is fundamental to preventing ossification, which has irreversibly affected TCP. 
Lastly, the \textbf{Sender Control of Acknowledgement Delay} feature allows the sender to control the delay of acknowledgment commands, reducing CPU load and improving performance.

Additionally, MASQUE plays a critical role in enabling proxy mechanisms for the QUIC protocol. 
MASQUE uses QUIC as a substrate protocol and addresses censorship issues by ensuring tunneled data is indistinguishable from conventional encrypted HTTP connections. 
Furthermore, the \textbf{CONNECT-UDP} method was proposed to mimic the TCP-only HTTP CONNECT method, enabling UDP-based tunneling. 
Building on this, \textbf{QUIC-Aware Proxying using CONNECT-UDP} was introduced to allow the proxying of arbitrary QUIC connections.

Beyond MASQUE, several other transport protocols utilize QUIC as a substrate protocol. 
For instance, \textbf{WebTransport} is an Application Programming Interface (API) developed by the IETF to integrate HTTP with QUIC, laying the foundation for further protocols. 
Specifically, \textbf{Http2Transport} and \textbf{Http3Transport} enable WebTransport peers to multiplex bidirectional data streams over HTTP/2 and HTTP/3 connections, respectively. 
In addition to the existing features of Http2Transport, Http3Transport introduces capabilities aimed at eliminating head-of-line (HOL) blocking. 
Lastly, \textbf{QuicTransport} is a minimalistic transport protocol that maps QUIC to arbitrary counterparts when applicable, focusing on maintaining low overhead and complexity.

In summary, QUIC was established to address the ossification of TCP by cleverly utilizing UDP as a substrate protocol to maximize compatibility. 
Over the years, the IETF has made significant efforts to expand QUIC's adoption and ensure its continued relevance.

