\section{Summary Paper 2}%
\label{sec:Summary Paper 2}

The paper by Grinnemo et al. presents a new transport protocol layer named the NEAT architecture (New, Evolutive API and Transport-layer architecture for the Internet) as a solution to the "ossification" of the Internet's architecture. \\
Since Internet's deployment, its layers have evolved to integrate new technologies (e.g TCP and UDP as standard transport protocols replacing NCP). However, its evolution has slowed down since its architecture is inflexible for different reasons (e.g widespread use of middleboxes). A rigid Internet architecture results in difficulties to introduce innovative protocols within its layers, and challenges least common situations such as peer-to-peer connectivity. Solutions have been suggested in recent years (e.g "virtual overlay techniques") using common protocols as a basis to be easily integrated (e.g QUIC transport protocol leaning on UDP and targeting web traffics). In general, these innovations only tackle specific points of the architecture and combining these approaches can be challenging so they fail to be adopted on a large scale. \\
As a solution, the authors propose NEAT, a transport system endowed with a User API that validates the requirements of "deployability, flexibility, evolvability and portability", all of which are necessary for a system to be both currently adoptable and "future-proof". Deployability and portability are achieved by the compatibility of the API with diverse network environments (e.g OS, network stacks). The flexibility and evolvability of the system stems from the functionalities of the API's five components. More specifically, the Framework component collects the communication preferences expressed by the applications and confronts them with the API's structure, enabling the Selection component to choose the appropriate settings for the transport layer (e.g "a transport endpoint"). These choices are combined with various requirements stated by the Policy component ensuring parameter suitability for the specific context (e.g "available interfaces"). Finally, the Transport component is responsible for the configuration and management of the chosen transport protocols, supported by the Signaling component for "advisory signaling" (e.g "communication with middleboxes, support for failover").\\
As a result, NEAT offers a promising solution to the rigidity of the Internet's architecture and tackles issues faced by the previous attempts of modernizing the Internet. However, the authors do not discuss any potential limitations even if there are some uncertainties (e.g configuration setting at runtime might lead to performance overhead) on their new system. Therefore, thorough testing and experimentation are crucial before widespread adoption of NEAT. 
