\section{Relation to the Course}%
\label{sec:Relation to the Course}

The three research papers explore advancements in transport layer protocols and relate to the course content in the context of modern networking needs.
The first paper \cite{langley2017quic} relies on \textbf{performance analysis and simulations} to highlight QUIC's performance gains (e.g reduced latency, higher throughput) compared to traditional TCP, similarly for the second paper \cite{grinnemo2016towards} to motivate the need for a flexible and evolvable architecture. Plus, both QUIC and NEAT likely leverage \textit{queuing theory} principles for congestion control, as queuing theory helps analyzing and optimizing the flow of data packets through networks, minimizing delays and maximizing throughput. Finally, the third paper \cite{kosek2021beyond} focuses on optimizing QUIC, presenting different features (e.g handshakes, stream multiplexing) aimed at improving QUIC's performance and efficiency through \textbf{information theory} concepts such as efficient data representation and error correction. 

