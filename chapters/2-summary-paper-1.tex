\section{Summary Paper 1}
\label{sec:Summary Paper 1}
With increasing growth in latency-sensitive web services the Google research group decided to design a new transport protocol independent of TCP, named the Quick UDP Internet Connections (QUIC). As to why they decided to avoid the TCP structure and focus on UDP is described in the paper as \textbf{Protocol and Implementation Entrenchment}, Handshake and Head-of-line Blocking Delays. We describe the ossification related problems of entrenchment in further detail with the following chapters of this seminar report,in this chapter we discuss:\\
\hspace*{1em}\textbf{Handshake Delays}: The round-trip handshake delays in TCP incur during connection setup before any data can be sent, affecting all connections. The most impacted however are short transfers which represent the majority of Internet traffic. \\
QUIC solves this problem by caching information (config and source-address token) about the server, thus any subsequent connection allows data to be sent immediately after clients handshake packet, without waiting for the servers response (0-RoundTripTime Handshake). \\
\hspace*{1em}\textbf{Head-of-line Blocking Delays}: HTTP/1.1 recommends limiting the number of connections from the client to the server, HTTP/2 goes as far as to recommending a single TCP connection to any server, both of these recommendations aim to reduce latency and overhead costs. However the TCP's bytestream abstraction fails to aid applications in their framing of communcations, resulting in "latency tax" \cite{langley2017quic} on the frames waiting on retransmissions of lost TCP segments.\\ 
To avoid this TCP's sequential delivery issue, QUIC supports multiple streams within a connection \cite{langley2017quic}. This ensures that only the streams with data on lost UDP packets are impacted, data on other streams continues to be reassembled and delivered without unnecessary retransmition waiting.

